\documentclass[12pt]{article}
\usepackage[utf8]{inputenc}

\usepackage{multirow}
\usepackage{tabularx}
\usepackage{enumerate}
\usepackage{lipsum}
\usepackage{float}
\usepackage[export]{adjustbox}
\usepackage{caption}
\usepackage{natbib}
\usepackage{graphicx}

\setlength{\marginparwidth}{2cm} 
\usepackage{todonotes}

\usepackage{listings}
\usepackage{xcolor}

\definecolor{codegreen}{rgb}{0,0.6,0}
\definecolor{codegray}{rgb}{0.5,0.5,0.5}
\definecolor{codepurple}{rgb}{0.58,0,0.82}
\definecolor{backcolour}{rgb}{0.95,0.95,0.92}

\graphicspath{ {./assets/} }

\usepackage[
    colorlinks=true,
    pdfborder={0 0 0},
    linkcolor=blue
]{hyperref}

\lstdefinestyle{mystyle}{
    backgroundcolor=\color{backcolour},   
    commentstyle=\color{codegreen},
    keywordstyle=\color{magenta},
    numberstyle=\tiny\color{codegray},
    stringstyle=\color{codepurple},
    basicstyle=\ttfamily\footnotesize,
    breakatwhitespace=false,         
    breaklines=true,                 
    captionpos=b,                    
    keepspaces=true,                 
    numbers=left,                    
    numbersep=5pt,                  
    showspaces=false,                
    showstringspaces=false,
    showtabs=false,                  
    tabsize=2
}

\lstset{style=mystyle}


\title{Bachelor Thesis Template}
\author{Pernille Lørup, Stephan D.}
\date{October 2020}
\setcounter{page}{0}

\begin{document}

\clearpage\maketitle
\thispagestyle{empty}

\pagebreak
\tableofcontents

\pagebreak
\section{Task 1}

For the first task we have found the documentation for writing a bachelor thesis on Moodle. Some PowerPoint slides are available from the previous semester introducing the requirements of the bachelor thesis. Below we will describe our findings.

\subsection{The formalities}

The maximum size of the bachelor project is 40 normal pages 
(2400 characters pr. page) pr. report +20 normal pages pr. student, i.e.:

\begin{itemize}
\item One student can hand-in up to 60 pages
\item Two students can hand-in up to 80 pages, etc
\end{itemize}

There is no absolute minimum, but teachers and
examiners experience it to be short if it is shorter
than 2/3 of the maximum.
\\
It can be written by groups as well as individual
students. 

\subsection{Relation to Internship}
It is recommended your bachelor project builds
on the experiences and the product of your
company work.
Often the internship report contains descriptions
of the company and the work you did. You can
reuse those parts if you state it clearly in the
bachelor report. 

\subsection{Content}
The bachelor report must focus on one or more of
the mandatory courses of the program:
\begin{itemize}
\item System Integration
\item Test
\item Databases for Developers
\item Large Systems Development\\
\end{itemize}

The bachelor project is typically centred on the
development of a product, the product being a
final developed and deployed product, a part of
such, or a prototype\\

The bachelor report should contain:
\begin{itemize}
\item A thorough description of the work done
\item A thorough evaluation and reflection of that work
\end{itemize}

\section{Task 2}
\subsection{Graphics}

% reference https://en.wikibooks.org/wiki/LaTeX/Floats,_Figures_and_Captions#Figures

\begin{figure}[H]
  \caption{This is a Caption above \label{figure:cap-above}}
  \includegraphics[width=\textwidth]{img}
\end{figure}

\begin{figure}[H]
  \includegraphics[width=\textwidth]{img}
  \caption{This is a Caption below  \label{figure:cap-below}}
\end{figure}

\begin{figure}[H]
    \captionsetup{justification=centering}
    \includegraphics[width=.6\textwidth, center]{img}
    \caption{This is a Caption centered  \label{figure:cap-center}}
\end{figure}

\begin{figure}[H]
    \captionsetup{justification=raggedright,singlelinecheck=false}
    \includegraphics[width=.6\textwidth, left]{img}
    \caption{This is a Caption to left  \label{figure:cap-left}}
\end{figure}

\begin{figure}[H]
    \captionsetup{justification=raggedleft,singlelinecheck=false}
    \includegraphics[width=.6\textwidth, right]{img}
    \caption{This is a Caption to right  \label{figure:cap-right}}
\end{figure}

\begin{figure}[H]
  \centering
  \begin{minipage}[b]{0.4\textwidth}
    \includegraphics[width=\textwidth]{img}
    \caption{Flower one.}
  \end{minipage}
  \hfill
  \begin{minipage}[b]{0.4\textwidth}
    \includegraphics[width=\textwidth]{img}
    \caption{Flower two.}
  \end{minipage}
  \label{figure:two-beside}
\end{figure}

\pagebreak
\subsection{References}

number for figure \ref{figure:cap-center} \\
page number for figure \pageref{figure:cap-center}

\subsection{Section}

Hello World!

\subsubsection{Subsection}

Structuring a document is easy!

\subsubsection{Subsubsection}

More text.

\paragraph{Paragraph}
\lipsum[1]\\

\subparagraph{Sub Paragraph}
\lipsum[1]\\

\paragraph{Paragraph}
\lipsum[1]

\subsection{Another section}

\subsubsection*{Non numeric Section}

\subsection{Lists}

\subsubsection*{Bullet Points}
\begin{itemize}
\item item a
\item item b
\item item c
\item item d
\end{itemize}

\subsubsection*{Alternative Points}
\begin{itemize}
\item[--] item a
\item[--] item b
\item[--] item c
\item[--] item d
\end{itemize}

\subsubsection*{Numbered List}
\begin{enumerate}
\item item a
\item item b
\item item c
\item item d
\end{enumerate}

\subsubsection*{Alternative Numbered List}
\begin{enumerate}[I]
\item item a
\item item b
\item item c
\item item d
\end{enumerate}

\subsubsection*{Alternative Numbered List}
\begin{enumerate}[a.]
\item item a
\item item b
\item item c
\item item d
\end{enumerate}

\subsection{Tables}
\subsubsection{Various horizontal alignments}
\begin{tabularx}{0.8\textwidth} {
  | >{\raggedright\arraybackslash}X
  | >{\centering\arraybackslash}X
  | >{\raggedleft\arraybackslash}X | }
 \hline
 item 11 & item 12 & item 13 \\
 \hline
 item 21  & item 22  & item 23  \\
\hline
\end{tabularx}

\subsubsection{Cell spanning}

\subsubsection{Multicolumn}
\begin{tabular}{|l|l|l|}
\hline
first column & second column & third column \\
\hline
\multicolumn{3}{|c|}{Cell spanning three columns} \\
\hline
\end{tabular}

\subsubsection{Multi row and alignment}

\begin{tabular}{|c|c|}
\hline
& T \\
\cline{2-2}
& X \\
\cline{2-2}
\multirow[c]{-3}[1]{*}{Multi row} & B \bigstrut \\
\hline
\end{tabular}



\subsubsubsection{Description and Labels}
\begin{table}[H]
\centering
\begin{tabular}{|c|c|}
\hline
\bfseries Column One & \bfseries Column Two\\
\hline
First data & 932\\ \hline
\end{tabular}
\caption{a simple table description}
\label{table:simple}
\end{table}

table number reference \ref{table:simple} \\
table page reference \pageref{table:simple}

\subsection{Code Listing}
\begin{verbatim}

function foo() {
    return console.log("Hello Latex!");
}

foo();

\end{verbatim}

\begin{lstlisting}[language=Python, caption=Python example]
def foo():
    return print("Hello Latex!")

foo()
\end{lstlisting}


\subsection{Bibliography}

\subsubsection{Book}
\begin{thebibliography}{1}
\bibitem{latexcompanion} 
Michel Goossens, Frank Mittelbach, and Alexander Samarin. 
\textit{The \LaTeX\ Companion}. 
Addison-Wesley, Reading, Massachusetts, 1993.

\subsubsection{Article}
\bibitem{einstein} 
Albert Einstein. 
\textit{Zur Elektrodynamik bewegter K{\"o}rper}. (German) 
[\textit{On the electrodynamics of moving bodies}]. 
Annalen der Physik, 322(10):891–921, 1905.

\subsubsection{Link}
\bibitem{knuthwebsite} 
Knuth: Computers and Typesetting,
\\\texttt{http://www-cs-faculty.stanford.edu/\~{}uno/abcde.html}
\end{thebibliography}

\subsection{ToDo Notes}
\todo{Write more to-do examples}

\end{document}
