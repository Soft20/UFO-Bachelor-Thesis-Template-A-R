\documentclass[12pt]{article}
\usepackage[utf8]{inputenc}

\usepackage{multirow}
\usepackage{tabularx}
\usepackage{enumerate}
\usepackage{lipsum}
\usepackage{float}
\usepackage[export]{adjustbox}
\usepackage{caption}
\usepackage{natbib}
\usepackage{graphicx}

\setlength{\marginparwidth}{2cm} 
\usepackage{todonotes}

\usepackage{listings}
\usepackage{xcolor}

\definecolor{codegreen}{rgb}{0,0.6,0}
\definecolor{codegray}{rgb}{0.5,0.5,0.5}
\definecolor{codepurple}{rgb}{0.58,0,0.82}
\definecolor{backcolour}{rgb}{0.95,0.95,0.92}

\graphicspath{ {./assets/} }

\usepackage[
    colorlinks=true,
    pdfborder={0 0 0},
    linkcolor=blue
]{hyperref}

\lstdefinestyle{mystyle}{
    backgroundcolor=\color{backcolour},   
    commentstyle=\color{codegreen},
    keywordstyle=\color{magenta},
    numberstyle=\tiny\color{codegray},
    stringstyle=\color{codepurple},
    basicstyle=\ttfamily\footnotesize,
    breakatwhitespace=false,         
    breaklines=true,                 
    captionpos=b,                    
    keepspaces=true,                 
    numbers=left,                    
    numbersep=5pt,                  
    showspaces=false,                
    showstringspaces=false,
    showtabs=false,                  
    tabsize=2
}

\lstset{style=mystyle}


\title{Task 2 - Bachelor Thesis Template}
\author{Adam Lass, Rasmus Helsgaun}
\date{October 2020}
\setcounter{page}{0}

\begin{document}

\clearpage\maketitle
\thispagestyle{empty}

\pagebreak
\tableofcontents

\pagebreak
\section{Graphics}

% reference https://en.wikibooks.org/wiki/LaTeX/Floats,_Figures_and_Captions#Figures

\begin{figure}[H]
  \caption{This is a Caption above \label{figure:cap-above}}
  \includegraphics[width=\textwidth]{img}
\end{figure}

\begin{figure}[H]
  \includegraphics[width=\textwidth]{img}
  \caption{This is a Caption below  \label{figure:cap-below}}
\end{figure}

\begin{figure}[H]
    \captionsetup{justification=centering}
    \includegraphics[width=.6\textwidth, center]{img}
    \caption{This is a Caption centered  \label{figure:cap-center}}
\end{figure}

\begin{figure}[H]
    \captionsetup{justification=raggedright,singlelinecheck=false}
    \includegraphics[width=.6\textwidth, left]{img}
    \caption{This is a Caption to left  \label{figure:cap-left}}
\end{figure}

\begin{figure}[H]
    \captionsetup{justification=raggedleft,singlelinecheck=false}
    \includegraphics[width=.6\textwidth, right]{img}
    \caption{This is a Caption to right  \label{figure:cap-right}}
\end{figure}

\begin{figure}[H]
  \centering
  \begin{minipage}[b]{0.4\textwidth}
    \includegraphics[width=\textwidth]{img}
    \caption{Flower one.}
  \end{minipage}
  \hfill
  \begin{minipage}[b]{0.4\textwidth}
    \includegraphics[width=\textwidth]{img}
    \caption{Flower two.}
  \end{minipage}
  \label{figure:two-beside}
\end{figure}

\pagebreak
\section{References}

number for figure \ref{figure:cap-center} \\
page number for figure \pageref{figure:cap-center}

\section{Section}

Hello World!

\subsection{Subsection}

Structuring a document is easy!

\subsection{Subsubsection}

More text.

\paragraph{Paragraph}
\lipsum[1]\\

\subparagraph{Sub Paragraph}
\lipsum[1]\\

\paragraph{Paragraph}
\lipsum[1]

\section{Another section}

\subsection*{Non numeric Section}

\section{Lists}

\subsection*{Bullet Points}
\begin{itemize}
\item item a
\item item b
\item item c
\item item d
\end{itemize}

\subsection*{Alternative Points}
\begin{itemize}
\item[--] item a
\item[--] item b
\item[--] item c
\item[--] item d
\end{itemize}

\subsection*{Numbered List}
\begin{enumerate}
\item item a
\item item b
\item item c
\item item d
\end{enumerate}

\subsection*{Alternative Numbered List}
\begin{enumerate}[I]
\item item a
\item item b
\item item c
\item item d
\end{enumerate}

\subsection*{Alternative Numbered List}
\begin{enumerate}[a.]
\item item a
\item item b
\item item c
\item item d
\end{enumerate}

\section{Tables}
\subsection{Various horizontal alignments}
\begin{tabularx}{0.8\textwidth} {
  | >{\raggedright\arraybackslash}X
  | >{\centering\arraybackslash}X
  | >{\raggedleft\arraybackslash}X | }
 \hline
 item 11 & item 12 & item 13 \\
 \hline
 item 21  & item 22  & item 23  \\
\hline
\end{tabularx}

\subsection{Cell spanning}

\subsection{Multicolumn}
\begin{tabular}{|l|l|l|}
\hline
first column & second column & third column \\
\hline
\multicolumn{3}{|c|}{Cell spanning three columns} \\
\hline
\end{tabular}

\subsection{Multi row and alignment}

\begin{tabular}{|c|c|}
\hline
& T \\
\cline{2-2}
& X \\
\cline{2-2}
\multirow[c]{-3}[1]{*}{Multi row} & B \bigstrut \\
\hline
\end{tabular}



\subsubsection{Description and Labels}
\begin{table}[H]
\centering
\begin{tabular}{|c|c|}
\hline
\bfseries Column One & \bfseries Column Two\\
\hline
First data & 932\\ \hline
\end{tabular}
\caption{a simple table description}
\label{table:simple}
\end{table}

table number reference \ref{table:simple} \\
table page reference \pageref{table:simple}

\section{Code Listing}
\begin{verbatim}

function foo() {
    return console.log("Hello Latex!");
}

foo();

\end{verbatim}

\begin{lstlisting}[language=Python, caption=Python example]
def foo():
    return print("Hello Latex!")

foo()
\end{lstlisting}


\section{Bibliography option 1}

Reference to a book: \cite{latexcompanion} \\
Reference to an article: \cite{einstein} \\
Reference to a website: \cite{knuthwebsite} \\

\bibliographystyle{plain}
\bibliography{references}

\section{Bibliography option 2}

\begin{thebibliography}{1}
\subsection{Book}
\bibitem{latexcompanion} 
Michel Goossens, Frank Mittelbach, and Alexander Samarin. 
\textit{The \LaTeX\ Companion}. 
Addison-Wesley, Reading, Massachusetts, 1993.

\subsection{Article}
\bibitem{einstein} 
Albert Einstein. 
\textit{Zur Elektrodynamik bewegter K{\"o}rper}. (German) 
[\textit{On the electrodynamics of moving bodies}]. 
Annalen der Physik, 322(10):891–921, 1905.

\subsection{Link}
\bibitem{knuthwebsite} 
Knuth: Computers and Typesetting,
\\\texttt{http://www-cs-faculty.stanford.edu/\~{}uno/abcde.html}
\end{thebibliography}


\section{ToDo Notes}
\todo{Write more to-do examples}

\end{document}
