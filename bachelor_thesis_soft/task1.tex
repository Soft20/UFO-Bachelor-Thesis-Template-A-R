\documentclass[12pt]{article}
\usepackage[utf8]{inputenc}

\usepackage{hyperref}
\hypersetup{
    colorlinks=true,
    linkcolor=blue,
    filecolor=magenta,      
    urlcolor=cyan,
}

\title{Task 1 - Bachelor Thesis Requirements}
\author{Adam Lass, Rasmus Helsgaun}
\date{October 2020}
\setcounter{page}{0}

\begin{document}

\clearpage\maketitle
\thispagestyle{empty}

\pagebreak
\tableofcontents

\pagebreak
\section{Introduction}

We have found the documentation for writing a bachelor thesis on Moodle and the CPHBusiness website. Some PowerPoint slides are available from the previous semester introducing the requirements of the bachelor thesis\cite{bachelorslides}. Furthermore we will be using the "Curriculum for the Bachelor’s Degree Programme in Software Development"\cite{curriculum}. Below we will describe our findings.

\section{The Formalities}

The maximum size of the bachelor project is 40 normal pages 
(2400 characters pr. page) pr. report +20 normal pages pr. student, i.e.:

\begin{itemize}
\item One student can hand-in up to 60 pages
\item Two students can hand-in up to 80 pages, etc
\end{itemize}

There is no absolute minimum, but teachers and
examiners experience it to be short if it is shorter
than 2/3 of the maximum.
\\
It can be written by groups as well as individual
students. 

\section{The Project}

The bachelor’s project must document the student’s understanding of and ability to reflect on the practices of the profession and the use of theory and methods in relation to a real-life problem. The problem statement, which must be central to the programme and profession, is formulated by the student, possibly in collaboration with a private or public company. The academy approves the problem statement.

\section{The Examination}

The exam is executed as an external examination that together with the other examinations in the education should document that the different learning objectives are met. The test consists of a written project and an oral examination that will be evaluated as one grade. The test can only be executed subsequent to a passed internship examination and the previous examinations in the education.

\section{Purpose}

In their bachelor’s project, the student must document the ability to work with a complex and practice-oriented issue in relation to a specific IT project, using an analytical and methodological basis.

\section{Learning Objectives}

The final bachelor project must demonstrate that the programme’s graduation level has been reached. The learning objectives for the study programme are consequently repeated below: 

\subsection{Knowledge}

The student must have knowledge of:
\begin{itemize}
\item The strategic role of testing in system development
\item Globalisation of software production
\item System architecture and its strategic importance for the company’s business
\item Applied theory and methodology and common technologies within the domain
\item Various database types and their applications
\end{itemize}

\subsection{Skills}

The student can:
\begin{itemize}
\item Integrate IT systems and develop systems that support future integration
\item Use contracts as a control and coordination mechanism in the development process
\item Assess and select database systems, and design, redesign and optimise databases
\item Plan and manage development processes involving many geographically separated project participants
\item Plan and implement testing for large IT systems
\end{itemize}

\subsection{Competencies}

The student can:
\begin{itemize}
\item Identify links between applied theory, methods and technology and reflect on their suitability in various situations
\item Engage in professional collaboration to develop large systems by applying common methods and technologies
\item Familiarise themselves with new technologies and standards for handling integration between systems
\item Through practice, develop their own competency profile from a primarily back-end developer profile to performing tasks as a system architect
\item Handle the establishment and realisation of a business and technologically appropriate architecture for large systems
\end{itemize}

\section{Relation to Internship}
It is recommended your bachelor project builds
on the experiences and the product of your
company work.
Often the internship report contains descriptions
of the company and the work you did. You can
reuse those parts if you state it clearly in the
bachelor report. 

\section{Content}
The bachelor report must focus on one or more of
the mandatory courses of the program:
\begin{itemize}
\item System Integration
\item Test
\item Databases for Developers
\item Large Systems Development\\
\end{itemize}

The bachelor project is typically centred on the
development of a product, the product being a
final developed and deployed product, a part of
such, or a prototype\\

The bachelor report should contain:
\begin{itemize}
\item A thorough description of the work done
\item A thorough evaluation and reflection of that work
\end{itemize}

\bibliographystyle{plain}
\bibliography{references}

\end{document}