\documentclass[12pt]{article}
\usepackage{packages}

\title{Bachelor Thesis Template}
\author{Pernille Lørup, Stephan D.}
\date{October 2020}
\setcounter{page}{0}

\begin{document}

\clearpage\maketitle
\thispagestyle{empty}

\pagebreak
\tableofcontents

\section{Graphics}

% reference https://en.wikibooks.org/wiki/LaTeX/Floats,_Figures_and_Captions#Figures

\begin{figure}[H]
  \caption{This is a Caption above \label{figure:cap-above}}
  \includegraphics[width=\textwidth]{img}
\end{figure}

\begin{figure}[H]
  \includegraphics[width=\textwidth]{img}
  \caption{This is a Caption below  \label{figure:cap-below}}
\end{figure}

\begin{figure}[H]
    \captionsetup{justification=centering}
    \includegraphics[width=.6\textwidth, center]{img}
    \caption{This is a Caption centered  \label{figure:cap-center}}
\end{figure}

\begin{figure}[H]
    \captionsetup{justification=raggedright,singlelinecheck=false}
    \includegraphics[width=.6\textwidth, left]{img}
    \caption{This is a Caption to left  \label{figure:cap-left}}
\end{figure}

\begin{figure}[H]
    \captionsetup{justification=raggedleft,singlelinecheck=false}
    \includegraphics[width=.6\textwidth, right]{img}
    \caption{This is a Caption to right  \label{figure:cap-right}}
\end{figure}

\begin{figure}[H]
  \centering
  \begin{minipage}[b]{0.4\textwidth}
    \includegraphics[width=\textwidth]{img}
    \caption{Flower one.}
  \end{minipage}
  \hfill
  \begin{minipage}[b]{0.4\textwidth}
    \includegraphics[width=\textwidth]{img}
    \caption{Flower two.}
  \end{minipage}
  \label{figure:two-beside}
\end{figure}

\pagebreak
\section{Image references}

number for figure \ref{figure:cap-center} \\
page number for figure \pageref{figure:cap-center}

\section{Section}

Hello World!

\subsection{Subsection}

Structuring a document is easy!

\subsubsection{Subsubsection}

More text.

\paragraph{Paragraph}
\lipsum[1]\\

\subparagraph{Sub Paragraph}
\lipsum[1]\\

\paragraph{Paragraph}
\lipsum[1]

\subsection{Another subsection}

\subsection*{Non numeric Section}

\pagebreak
\section{Lists}

\subsection*{Bullet Points}
\begin{itemize}
\item item a
\item item b
\item item c
\item item d
\end{itemize}

\subsection*{Alternative Points}
\begin{itemize}
\item[--] item a
\item[--] item b
\item[--] item c
\item[--] item d
\end{itemize}

\subsection*{Numbered List}
\begin{enumerate}
\item item a
\item item b
\item item c
\item item d
\end{enumerate}

\subsection*{Alternative Numbered List}
\begin{enumerate}[I]
\item item a
\item item b
\item item c
\item item d
\end{enumerate}

\subsection*{Alternative Numbered List}
\begin{enumerate}[a.]
\item item a
\item item b
\item item c
\item item d
\end{enumerate}

\pagebreak

\section{Tables}
\subsection{Various horizontal alignments}
\begin{tabularx}{0.8\textwidth} {
  | >{\raggedright\arraybackslash}X
  | >{\centering\arraybackslash}X
  | >{\raggedleft\arraybackslash}X | }
 \hline
 item 1 & item 2 & item 3 \\
 \hline
 item 10  & item 20  & item 30  \\
\hline
\end{tabularx}

\subsection{Cell spanning}

\subsubsection{Multicolumn}
\begin{tabular}{|l|l|l|}
\hline
first column & second column & third column \\
\hline
\multicolumn{3}{|c|}{Cell spanning three columns} \\
\hline
\end{tabular}

\subsubsection{Multi row and alignment}

\begin{tabular}{|c|c|}
\hline
& A \\
\cline{2-2}
& B \\
\cline{2-2}
\multirow[c]{-3}[1]{*}{Multi row} & C \bigstrut \\
\hline
\end{tabular}


\subsection{Description and Labels}
\begin{table}[H]
\centering
\begin{tabular}{|c|c|}
\hline
\bfseries Column One & \bfseries Column Two\\
\hline
First data & Second data\\ \hline
\end{tabular}
\caption{a simple table description}
\label{table:simple}
\end{table}

\noindent table number reference \ref{table:simple} \\
table page reference \pageref{table:simple}

\pagebreak

\section{Code Listing}
\begin{verbatim}

function foo() {
    return console.log("Hello Latex!");
}

foo();

\end{verbatim}

\begin{lstlisting}[language=Python, caption=Python example]
def foo():
    return print("Hello Latex!")

foo()
\end{lstlisting}

\pagebreak 

\section{Bibliography}

This is a referencing text to a book \cite{latexcompanion} \\
This is an article reference \cite{einstein} \\
This is a reference to a website \cite{knuthwebsite} \\

\bibliographystyle{plain}
\bibliography{references}

\section{ToDo Notes}
\todo{Write more to-do examples}
\todo{Write a lot more to-do examples}

\end{document}
